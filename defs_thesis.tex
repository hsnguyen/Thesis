% Copyright (C) 2004 Paul Cochrane
%
% This program is free software; you can redistribute it and/or
% modify it under the terms of the GNU General Public License
% as published by the Free Software Foundation; either version 2
% of the License, or (at your option) any later version.
%
% This program is distributed in the hope that it will be useful,
% but WITHOUT ANY WARRANTY; without even the implied warranty of
% MERCHANTABILITY or FITNESS FOR A PARTICULAR PURPOSE.  See the
% GNU General Public License for more details.
%
% You should have received a copy of the GNU General Public License
% along with this program; if not, write to the Free Software
% Foundation, Inc., 59 Temple Place - Suite 330, Boston, MA  02111-1307, USA.

\newcommand {\tbf}[1] {\textbf{#1}}
\newcommand {\tit}[1] {\textit{#1}}
\newcommand {\tmd}[1] {\textmd{#1}}
\newcommand {\trm}[1] {\textrm{#1}}
\newcommand {\tsc}[1] {\textsc{#1}}
\newcommand {\tsf}[1] {\textsf{#1}}
\newcommand {\tsl}[1] {\textsl{#1}}
\newcommand {\ttt}[1] {\texttt{#1}}
\newcommand {\tup}[1] {\textup{#1}}

\newcommand {\mbf}[1] {\mathbf{#1}}
\newcommand {\mmd}[1] {\mathmd{#1}}
\newcommand {\mrm}[1] {\mathrm{#1}}
\newcommand {\msc}[1] {\mathsc{#1}}
\newcommand {\msf}[1] {\mathsf{#1}}
\newcommand {\msl}[1] {\mathsl{#1}}
\newcommand {\mtt}[1] {\mathtt{#1}}
\newcommand {\mup}[1] {\mathup{#1}}

%My thesis abbr.
\newcommand{\npscarf}{$\mathtt{npScarf}$}
\newcommand{\npscarfg}{$\mathtt{npScarf\_wag}$}
\newcommand{\npreader}{$\mathtt{npReader}$}
\newcommand{\npanalysis}{$\mathtt{npAnalysis}$}
\newcommand{\npbarcode}{$\mathtt{npBarcode}$}
\newcommand{\npgraph}{$\mathtt{npGraph}$}
\newcommand{\canu}{$\mathtt{Canu}$}
\newcommand{\unicycler}{$\mathtt{Unicycler}$}
\newcommand{\spades}{$\mathtt{SPAdes}$}
\newcommand{\albacore}{$\mathtt{Albacore}$}
\newcommand{\racon}{$\mathtt{Racon}$}
\newcommand{\metrichor}{$\mathtt{Metrichor}$}
\newcommand{\minimap}{$\mathtt{minimap2}$}
\newcommand{\miniasm}{$\mathtt{miniasm}$}
\newcommand{\bwa}{$\mathtt{BWA\text{-}MEM}$}

\newcommand{\ec}{\emph{E.~coli}}
\newcommand{\sce}{\emph{S.~cerevisiae}}
\newcommand{\kp}{\emph{K.~pneumoniae}} 

\newcommand{\IE}{\emph{i.e.}}
\newcommand{\EG}{\emph{e.g.}}
\newcommand{\review}[1]{\textcolor{red}{#1}}

\newcommand {\figwidth} {100mm}
\newcommand {\Ref}[1] {Reference~\cite{#1}}
\newcommand {\Sec}[1] {Section~\ref{#1}}
\newcommand {\App}[1] {Appendix~\ref{#1}}
\newcommand {\Chap}[1] {Chapter~\ref{#1}}
\newcommand {\etal} {\emph{~et~al.}}
\newcommand {\bul} {$\bullet$ }   % bullet
\newcommand {\fig}[1] {Figure~\ref{#1}}   % references Figure x
\newcommand {\imp} {$\Rightarrow$}   % implication symbol (default)
\newcommand {\impt} {$\Rightarrow$}   % implication symbol (text mode)
\newcommand {\impm} {\Rightarrow}   % implication symbol (math mode)
\newcommand {\vect}[1] {\mathbf{#1}}
\newcommand {\hvect}[1] {\hat{\mathbf{#1}}}
\newcommand {\del} {\partial}
\newcommand {\eqn}[1] {Equation~(\ref{#1})}
\newcommand {\tab}[1] {Table~\ref{#1}} % references Table x
\newcommand {\half} {\frac{1}{2}}
\newcommand {\ten}[1] {\times10^{#1}}
\newcommand {\bra}[2] {\mbox{}_{#2}\langle #1 |}
\newcommand {\ket}[2] {| #1 \rangle_{#2}}
\newcommand {\Bra}[2] {\mbox{}_{#2}\left.\left\langle #1 \right.\right|}
\newcommand {\Ket}[2] {\left.\left| #1 \right.\right\rangle_{#2}}
\newcommand {\im} {\mathrm{Im}}
\newcommand {\re} {\mathrm{Re}}
\newcommand {\braket}[4] {\mbox{}_{#3}\langle #1 | #2 \rangle_{#4}}
\newcommand {\dotprod}[4] {\mbox{}_{#3}\langle #1 | #2 \rangle_{#4}}
\newcommand {\trace}[1] {\text{tr}\left(#1\right)}

% spell things correctly
\newenvironment{centre}{\begin{center}}{\end{center}}
\newenvironment{itemise}{\begin{itemize}}{\end{itemize}}

\usepackage{epigraph}
\setlength\epigraphwidth{.5\textwidth}
\setlength\epigraphrule{0pt}

\usepackage{pdfpages} 

\usepackage{acronym} 
\usepackage[titletoc]{appendix}
\usepackage{mathtools}
\usepackage{play}
\usepackage[grey,times]{quotchap}
\usepackage{makeidx}
\usepackage{xcolor,colortbl}
\usepackage{longtable}
\usepackage{booktabs}
\usepackage{hyperref}
\usepackage{pdflscape}
\usepackage{afterpage}
%%%%% set up the bibliography style
\bibliographystyle{uqthesis}  % uqthesis bibliography style file, made
			      % with makebst

%%%%% optional packages
\usepackage[square,comma,numbers,sort&compress]{natbib}
		% this is the natural sciences bibliography citation
		% style package.  The options here give citations in
		% the text as numbers in square brackets, separated by
		% commas, citations sorted and consecutive citations
		% compressed
		% output example: [1,4,12-15]

\usepackage[nottoc]{tocbibind}
				% allows the table of contents, bibliography
				% and index to be added to the table of
				% contents if desired, the option used
				% here specifies that the table of
				% contents is not to be added.
				% tocbibind needs to be after natbib
				% otherwise bits of it get trampled.

\usepackage{amsmath,amsfonts,amssymb} % this is handy for mathematicians and physicists
			      % see http://www.ams.org/tex/amslatex.html

% \usepackage{showkeys} % this shows what labels you are using for cross
		      % references

\usepackage{graphicx} % standard graphics package for inclusion of
		      % images and eps files into LaTeX document

\usepackage{multirow} 

% For pseudocode
\usepackage[linesnumbered,boxed,ruled,vlined]{algorithm2e}
\newcommand\mycommfont[1]{\footnotesize\ttfamily\textcolor{blue}{#1}}
\SetCommentSty{mycommfont}
% For code embedding (Bash, Java...)
\usepackage{listings}
\lstset{basicstyle=\ttfamily,
  showstringspaces=false,
  commentstyle=\color{red},
  keywordstyle=\color{blue}
}

\usepackage{tocloft}

\usepackage{float}
\usepackage[caption = false]{subfig}
% this code hacked from that of R Chandrasekhar from UWA
\newif\ifpdf
\ifx\pdfoutput\undefined
	\pdffalse    % we are not running pdfLaTeX
\else
	\pdfoutput=1 % we are running pdfLaTeX
	\pdftrue
\fi

\ifpdf
	\DeclareGraphicsExtensions{.pdf}  % this command defined in graphicx
	\pdfcompresslevel=9  % 0: no compression, 9: highest compression
			     % or, set compress_level 9 in file pdftex.cfg
\else
	\DeclareGraphicsExtensions{.ps}
\fi

% put in an index?
% \makeindex

%set numbering for subsubsection
\setcounter{secnumdepth}{3}

\definecolor{Gray}{gray}{0.9}
\newcommand{\cir}{$^\ast$}
\newcommand{\bres}[1]{{\bf #1}}